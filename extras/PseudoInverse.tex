
\documentclass{article}

\usepackage{hyperref}
\usepackage{amsmath}
\usepackage{amssymb}

\newtheorem{theorem}{Theorem}

\begin{document}

\title{The 2 by 2 Real Pseudo Inverse}
\author{John Mount}
\date{2019-01-07}




\maketitle


This is a brief note on the math behind the direct
PRESS statistic calculation\footnote{\url{http://www.win-vector.com/blog/2014/09/estimating-generalization-error-with-the-press-statistic/}} 
found in the RcppDynProg package\footnote{\url{https://github.com/WinVector/RcppDynProg}}.

The actual `C++` code\footnote{\url{https://github.com/WinVector/RcppDynProg/blob/master/src/xlin_fits.cpp}} is a bit ugly and intimidating.  That is because we are using a verbose scalar notation to represent matrix concepts.  In matrix notation we are solving a linear system by inverting a two by two matrix.\footnote{Yes, there are the usual admonitions that one should not invert a matrix to solve a linear system, but for systems this small they do not apply.}  We are in turn inverting the two by two matrix by exploiting the following well know rule of how to invert a two by two matrix.


If $a d - b c$ is not zero then:

\[ 
\begin{pmatrix} a & b \\ c & d \end{pmatrix}^{-1}
=
\frac{1}{a d - b c}
\begin{pmatrix} d & -b \\ -c & a \end{pmatrix}
\]

Throughout $a$, $b$, $c$, and $d$ all real scalars.

This can be re-written as the following general relation.

\[ 
\begin{pmatrix} a & b \\ c & d \end{pmatrix}
\begin{pmatrix} d & -b \\ -c & a \end{pmatrix}
=
(a d - b c) 
\begin{pmatrix} 1 & 0 \\ 0 & 1 \end{pmatrix}
\]

The above can be directly checked just by applying the rules for multiplying matrices to the left two matrices.  This has a particularly pleasant presentation if we recognize that $a d - b c$ is  the determinant of the left matrix and use the traditional vertical bar determinant notation.

\[ 
\begin{pmatrix} a & b \\ c & d \end{pmatrix}
\begin{pmatrix} d & -b \\ -c & a \end{pmatrix}
=
\begin{vmatrix} a & b \\ c & d \end{vmatrix}
\begin{pmatrix} 1 & 0 \\ 0 & 1 \end{pmatrix}
\]



Now there is an issue of what to do when $a d - b c$ is zero.  For our implementation we apply Tikhonov regularization\footnote{\url{https://en.wikipedia.org/wiki/Tikhonov_regularization}} which (barring the exact numeric coincidence of minus two times the expected value of the independent variable equaling our regularization constant) is going to be non-singular.  For our actual application, we could simply switch degenerate situations to the out-of sample mean implementation\footnote{\url{https://github.com/WinVector/RcppDynProg/blob/master/src/const_costs.cpp}}.

But, for fun, let's play with the math a bit.

There is an additional lesser known algebraic relation for two by two matrices.


\[ 
\begin{pmatrix} a & b \\ c & d \end{pmatrix}
\begin{pmatrix} a & c \\ b & d \end{pmatrix}
\begin{pmatrix} a & b \\ c & d \end{pmatrix}
=
(a^2 + b^2 + c^2 + d^2) 
\begin{pmatrix} a & b \\ c & d \end{pmatrix} 
+
(a d - b c)
\begin{pmatrix} -d & c \\ b & -a \end{pmatrix} 
\]

Or (using transpose, matrix squared Frobenius norm, and determinant notation):

\begin{theorem}
For any real 2 by 2 matrix
$\begin{pmatrix} a & b \\ c & d \end{pmatrix}$ we have:


\[
\begin{pmatrix} a & b \\ c & d \end{pmatrix}
\begin{pmatrix} a & b \\ c & d \end{pmatrix}^{\top}
\begin{pmatrix} a & b \\ c & d \end{pmatrix}
=
\begin{Vmatrix} a & b \\ c & d \end{Vmatrix}_2^2
\begin{pmatrix} a & b \\ c & d \end{pmatrix} 
+
\begin{vmatrix} a & b \\ c & d \end{vmatrix}
\begin{pmatrix} -d & c \\ b & -a \end{pmatrix} 
\].
The superscript "top" denoting the transpose operation, the $||.||^2_2$ denoting sum of squares norm, and the single $|.|$ denoting determinant.
\\ $\square$
\label{thm:fmla}
\end{theorem}


This means, if $a d - b c$ is zero then:


\[ 
\begin{pmatrix} a & b \\ c & d \end{pmatrix}
\begin{pmatrix} a & b \\ c & d \end{pmatrix}^{\top}
\begin{pmatrix} a & b \\ c & d \end{pmatrix}
=
\begin{Vmatrix} a & b \\ c & d \end{Vmatrix}_2^2
\begin{pmatrix} a & b \\ c & d \end{pmatrix}
\]


Once we [confirm the above relation\footnote{\url{https://github.com/WinVector/RcppDynProg/blob/master/extras/PseudoInverse.ipynb}} we can also confirm that if $a^2 + b^2 + c^2 + d^2$ is not zero, then the following matrix:

\[ 
\begin{pmatrix} a & b \\ c & d \end{pmatrix}^{+}
=
\frac{1}{a^2 + b^2 + c^2 + d^2}
\begin{pmatrix} a & c \\ b & d \end{pmatrix}
\]

satisfies all of the conditions for being the Moore-Penrose inverse\footnote{\url{https://en.wikipedia.org/wiki/Moore–Penrose_inverse}} (or pseudo-inverse) of our original matrix.  The superscript-plus denoting the Moore-Penrose inverse operation.

Or in transpose notation:

\[ 
\begin{pmatrix} a & b \\ c & d \end{pmatrix}^{+}
=
\frac{1}{a^2 + b^2 + c^2 + d^2}
\begin{pmatrix} a & b \\ c & d \end{pmatrix}^{\top}
\]

This is called the pseudo-inverse because it acts like an inverse, even for non-invertible matrices.  For $A^{+}$ to be
a More-Penrose inverse we must confirm it obeys the following relations:

\begin{align*}
A A^{+} A &= A \\
A^{+} A A^{+} &= A^{+} \\
(A A^{+})^{\top} &= A A^{+} \\
(A^{+} A)^{\top} &= A^{+} A
\end{align*}

Theorem \ref{thm:fmla} lets us check the first relation, the other follow quickly as our $A^{+}$ is a simple scalar multiple of the transpose.\footnote{With the same scaling for both $A^{+}$ and $(A^{\top})^{+}$.}

All of the above check relations would be true for a classic inverse.  We can think of $A^{+}$ as almost canceling a single $A$ to the left or the $A$ to the right.


\begin{theorem}
For any real 2 by 2 matrix $\begin{pmatrix} a & b \\ c & d \end{pmatrix}$

The  Moore-Penrose inverse is:


\begin{itemize}


\item When $a d - b c$ is not zero:

\[
\frac{1}{a d - b c}
\begin{pmatrix} d & -b \\ -c & a \end{pmatrix} 
\]

\item When $a d - b c$ is zero and $a^2 + b^2 + c^2 + d^2$ is not zero:

\[ 
\frac{1}{a^2 + b^2 + c^2 + d^2}
\begin{pmatrix} a & c \\ b & d \end{pmatrix}
\]

\item Otherwise:

\[ 
\begin{pmatrix} 0 & 0 \\ 0 & 0 \end{pmatrix}
\]
\end{itemize}
.
\\ $\square$
\end{theorem}

For general matrices the situation is much more complicated. 

The wealth of symmetries and relations is really kind of neat.




\end{document}
